\documentclass[11pt,letterpaper]{article}

\usepackage[margin=0.9in]{geometry}
\usepackage{amsmath,amssymb}
\usepackage{booktabs}
\usepackage{array}
\usepackage{xcolor}
\usepackage{enumitem}
\usepackage{hyperref}
\usepackage{microtype}
\usepackage{fancyvrb}
\usepackage{float}
\usepackage{colortbl}

\hypersetup{colorlinks=true, linkcolor=blue!60!black, urlcolor=blue!60!black}

\definecolor{bccgreen}{HTML}{e8f5e8}
\definecolor{scyellow}{HTML}{fff8e0}
\definecolor{fccred}{HTML}{fde8e8}
\definecolor{headbg}{HTML}{e8e8f0}
\definecolor{darkblue}{HTML}{1a1a2e}

\pagestyle{empty}

\newcommand{\ainv}{\alpha^{-1}}
\newcommand{\ppb}{\,\mathrm{ppb}}

\setlength{\parskip}{4pt}
\setlength{\parindent}{0pt}

\begin{document}

% -- Title --
\begin{center}
  {\Large\bfseries\color{darkblue}
    Numerical Coincidences from a BCC Lattice Framework}\\[4pt]
  {\itshape A concise summary for discussion --- not a claim of proof}\\[2pt]
  {\small Alan Garcia \quad|\quad February 2026 \quad|\quad Independent investigation}
\end{center}
\vspace{-2pt}
\hrule
\vspace{6pt}

% -- 1 --
\subsection*{1.\ What This Is}

I am an independent researcher exploring a lattice field theory framework
in which the fine structure constant, the proton-electron mass ratio, and
the muon-electron mass ratio emerge from the spectral geometry of the BCC
lattice Dirac operator. The framework takes two inputs---the BCC
coordination number $n = 8$ and the transcendental $\pi$---and produces
values for all three constants with zero free parameters. I am writing to
ask whether you can see a trivial explanation for the coincidences below,
or whether the lattice selection mechanism has structural content.

% -- 2 --
\subsection*{2.\ Generator Functions from $\mathrm{SO}(3)$ Representation Theory}

Four functions of the coordination number $n$ arise from the
representation theory of $\mathrm{SO}(3)$:

\begin{equation*}
  \tau(n) = n(2n{+}1) + 1 \qquad
  \sigma(n) = n(2n{+}1) \qquad
  \rho(n) = n + 1 \qquad
  \mu(n) = \tfrac{3}{2}\,\sigma\,\rho
\end{equation*}

At $n = 8$: $\tau = 137$, $\sigma = 136$, $\rho = 9$, $\mu = 1836$.
Here $\tau(n) = \dim\bigl(\Lambda^2 D(n) \oplus D(0)\bigr)$, where
$\Lambda^2 D(n)$ is the antisymmetric square of the spin-$n$
representation, and $\sigma(n) = \binom{2n+1}{2}$ counts the pairwise
couplings among $2n{+}1$ magnetic substates.

% -- 3 --
\subsection*{3.\ Result 1: The Fine Structure Constant}

$\ainv$ is obtained from a cubic equation interpreted as a Dyson equation
$G^{-1} = G_0^{-1} - \Sigma(G)$ for the dressed inverse coupling:

\begin{equation*}
  \boxed{\ainv = B + \frac{1}{(n{+}2)\,B^2}}, \qquad
  B = \tau + \frac{c_1}{\tau} - \frac{1}{2\tau^2}
    - \frac{1}{(n{-}1)\,\tau^3}
    + \frac{c_4}{\tau^4}
\end{equation*}

The base $B$ is a perturbative self-energy expansion on the BCC lattice
with coupling $g^2 = 1/2$ and Dirac multiplicity $N_D = 4$. The
coefficients have identified spectral origins:
\begin{itemize}[nosep,leftmargin=1.5em]
  \item $c_1 = \pi^2/2$ \quad Brillouin zone momentum integral
    $\langle|\mathbf{q}|^2\rangle_{\mathrm{BZ}}$
  \item $c_2 = -\tfrac{1}{2}$ \quad Spectral mean:
    $-d/\langle D^2\rangle$
  \item $c_3 = -1/(n{-}1) = -\tfrac{1}{7}$ \quad Spectral variance
  \item $c_4 = \tfrac{2\rho}{2n{+}1}\,\pi^3 = \tfrac{18}{17}\,\pi^3$
    \quad Spectral skewness $\times$ zone integral
\end{itemize}

\begin{table}[H]
\centering\small
\begin{tabular}{lc}
\toprule
Lattice framework prediction & $137.035\,999\,177$ \\
CODATA 2022 & $137.035\,999\,177(21)$ \\
Agreement & $< 0.001\ppb$ \\
\bottomrule
\end{tabular}
\end{table}

% -- 4 --
\subsection*{4.\ Result 2: The Proton-Electron Mass Ratio}

From the same generator functions, with $\alpha$ computed self-consistently.
The formula appears naturally in $n$-explicit factored form:

\begin{equation*}
  \boxed{\frac{m_p}{m_e}
    = \mu\bigl(1 + 2(2n{+}1)\,\pi\alpha^3\bigr)
    + \Bigl(1 + \frac{1}{2n}\Bigr)\pi^2\alpha(1{-}\alpha^2)
    - \alpha^2}
\end{equation*}

The denominators in the original notation ($\pi\mu\sigma\alpha^3/4$ and
$(2n{+}1)\pi^2\alpha/16$) are functions of the coordination number:
$4 = n/2$ and $16 = 2n$ at $n = 8$. The corrections decompose as:
\begin{itemize}[nosep,leftmargin=1.5em]
  \item \textbf{Vertex:} $\mu$ is dressed by a factor
    $1 + 2(2n{+}1)\pi\alpha^3$, where $2(2n{+}1) = 2\dim D(n)$.
  \item \textbf{VP:} A universal piece $\pi^2\alpha(1{-}\alpha^2)$ plus
    a lattice correction of relative size $1/(2n)$, vanishing as
    $n \to \infty$ (continuum limit).
  \item \textbf{Self-intersection:} $-\alpha^2$, universal.
\end{itemize}

\begin{table}[H]
\centering\small
\begin{tabular}{lc}
\toprule
Lattice framework prediction & $1836.152\,673\,5$ \\
CODATA 2022 & $1836.152\,673\,426(32)$ \\
Agreement & $< 0.03\ppb$ \\
\bottomrule
\end{tabular}
\end{table}

% -- 5 --
\subsection*{5.\ Result 3: The Muon-Electron Mass Ratio}

The muon is modeled as a binary defect excitation (two-node) on the same
lattice, in contrast to the proton's ternary defect (three-node). Its
formula follows the same correction taxonomy but with $d$ (spatial
dimension) replacing $n$ (coordination number) as the structural
parameter:

\begin{equation*}
  \boxed{\frac{m_\mu}{m_e}
    = \frac{d}{2}\bigl(\tau + \pi\alpha(1{-}\alpha^2)\bigr)
    + \frac{c_1}{4}
    + \frac{d}{2}\,\pi\sigma\alpha^3
    - \alpha^2}
\end{equation*}

The proton's vertex prefactor is $2/n$ (per-bond weight on a lattice with
$n$ neighbors); the muon's is $d/2$ (dimensional weight in $d$-dimensional
space). The muon vertex coefficient $(d/2)\,\pi\,\sigma$ equals the proton
vertex coefficient times $2d/\mu = 6/1836 = 1/306$.

\begin{table}[H]
\centering\small
\begin{tabular}{lc}
\toprule
Lattice framework prediction & $206.768\,282\,5$ \\
CODATA 2022 & $206.768\,282\,7(46)$ \\
Agreement & $1.1\ppb$ \quad ($0.05\sigma$) \\
\bottomrule
\end{tabular}
\end{table}

% -- 6 --
\subsection*{6.\ Lattice Selection: Why BCC?}

Tested against SC ($n = 6$) and FCC ($n = 12$). The tree-level mass
generator $\mu(n)$ is the discriminator:

\begin{table}[H]
\centering\small
\begin{tabular}{cccccc}
\toprule
\textbf{Lattice} & $n$ & $\mu(n)$ & $\ainv$ \textbf{error}
  & \textbf{Mass error} & \textbf{Status} \\
\midrule
\rowcolor{bccgreen} BCC & 8 & 1836 & $< 0.001\ppb$ & $< 0.03\ppb$
  & Passes all \\
\rowcolor{scyellow} SC & 6 & 819 & $9.9\ppb$ & $-55.4\%$
  & Fails mass \\
\rowcolor{fccred} FCC & 12 & 5850 & $142.7\ppb\;(6.8\sigma)$ & $+218.6\%$
  & Fails both \\
\bottomrule
\end{tabular}
\end{table}

SC's $\ainv$ closeness is explained by $\langle D^2\rangle = 2d$ universality
for cubic lattices (first two loop corrections identical). Discrimination enters
at three loops via spectral variance and, decisively, through $\mu(n)$: no
perturbative correction bridges SC's 55\% mass deficit.

% -- 7 --
\subsection*{7.\ Questions for the Reader}

\begin{enumerate}[label=(\alph*),nosep,leftmargin=1.5em]
  \item Is there a trivial or known reason why polynomials of 8 produce
    integers close to 137 and 1836?
  \item Does $g^2 = 1/2$, $N_D = 4$ on BCC correspond to a recognized
    lattice action?
  \item The proton vertex denominator is $n/2$ and the VP denominator is
    $2n$. Does this $n$-dependence have precedent in lattice perturbation
    theory?
  \item The proton's structural parameter is $n$ (coordination); the
    muon's is $d$ (dimension). Does this topological/geometric distinction
    correspond to anything in defect field theory?
  \item Is the SC/BCC near-degeneracy for $\ainv$
    (via $\langle D^2\rangle = 2d$) interesting or expected?
\end{enumerate}

\medskip
I would be grateful for any feedback, including identification of a trivial
explanation. Full paper (\LaTeX) available on request.

\vspace{6pt}
\hrule
\vspace{4pt}

% -- Appendix --
\subsection*{Appendix: Verification Code (Python)}

\begin{Verbatim}[fontsize=\footnotesize]
import numpy as np
def tau(n): return n*(2*n+1)+1
def sigma(n): return n*(2*n+1)
def mu(n): return 1.5*sigma(n)*(n+1)
N = 8; d = 3
TAU, SIGMA, RHO, MU = tau(N), sigma(N), N+1, mu(N)
c1 = np.pi**2 / 2
c4 = (2*RHO/(2*N+1)) * np.pi**3
B = TAU + c1/TAU - 1/(2*TAU**2) - 1/((N-1)*TAU**3) + c4/TAU**4
alpha_inv = B + 1/((N+2)*B**2); alpha = 1/alpha_inv
mass = MU*(1+2*(2*N+1)*np.pi*alpha**3) \
       + (1+1/(2*N))*np.pi**2*alpha*(1-alpha**2) - alpha**2
muon = (d/2)*(TAU+np.pi*alpha*(1-alpha**2)) \
       + c1/4 + (d/2)*np.pi*SIGMA*alpha**3 - alpha**2
print(f"alpha^-1  = {alpha_inv:.12f}")  # 137.035999177
print(f"m_p/m_e   = {mass:.9f}")        # 1836.152673485
print(f"m_mu/m_e  = {muon:.10f}")       # 206.7682824754
\end{Verbatim}

\vspace{4pt}
\hrule
\vspace{3pt}
\begin{center}
{\footnotesize\color{gray}
  Contact: \href{mailto:alan.javier.garcia@gmail.com}{alan.javier.garcia@gmail.com} \quad|\quad Full paper and working notes available on request}
\end{center}

\end{document}
