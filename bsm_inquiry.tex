\documentclass[11pt,letterpaper]{article}

\usepackage[margin=0.9in]{geometry}
\usepackage{amsmath,amssymb}
\usepackage{booktabs}
\usepackage{array}
\usepackage{xcolor}
\usepackage{enumitem}
\usepackage{hyperref}
\usepackage{microtype}
\usepackage{fancyvrb}
\usepackage{float}
\usepackage{colortbl}

\hypersetup{colorlinks=true, linkcolor=blue!60!black, urlcolor=blue!60!black}

\definecolor{bccgreen}{HTML}{e8f5e8}
\definecolor{scyellow}{HTML}{fff8e0}
\definecolor{fccred}{HTML}{fde8e8}
\definecolor{headbg}{HTML}{e8e8f0}
\definecolor{darkblue}{HTML}{1a1a2e}

\pagestyle{empty}

\newcommand{\ainv}{\alpha^{-1}}
\newcommand{\ppb}{\,\mathrm{ppb}}

\setlength{\parskip}{4pt}
\setlength{\parindent}{0pt}

\begin{document}

% ── Title ──────────────────────────────────────────────────────────────
\begin{center}
  {\Large\bfseries\color{darkblue}
    Numerical Coincidences from a BCC Lattice Framework}\\[4pt]
  {\itshape A concise summary for discussion --- not a claim of proof}\\[2pt]
  {\small Alan Garcia \quad|\quad February 2026 \quad|\quad Independent investigation}
\end{center}
\vspace{-2pt}
\hrule
\vspace{6pt}

% ── 1 ──────────────────────────────────────────────────────────────────
\subsection*{1.\ What This Is}

I am an independent researcher exploring a lattice field theory framework
in which the fine structure constant and the proton-electron mass ratio
emerge from the spectral geometry of the BCC lattice Dirac operator. The
framework takes two inputs---the BCC coordination number $n = 8$ and the
transcendental $\pi$---and produces values for both constants with zero
free parameters. I am writing to ask whether you can see a trivial
explanation for the coincidences below, or whether the lattice selection
mechanism has structural content.

% ── 2 ──────────────────────────────────────────────────────────────────
\subsection*{2.\ Generator Functions from $\mathrm{SO}(3)$ Representation Theory}

Four functions of the coordination number $n$ arise from the
representation theory of $\mathrm{SO}(3)$:

\begin{equation*}
  \tau(n) = n(2n{+}1) + 1 \qquad
  \sigma(n) = n(2n{+}1) \qquad
  \rho(n) = n + 1 \qquad
  \mu(n) = \tfrac{3}{2}\,\sigma\,\rho
\end{equation*}

At $n = 8$: $\tau = 137$, $\sigma = 136$, $\rho = 9$, $\mu = 1836$.
Here $\tau(n) = \dim\bigl(\Lambda^2 D(n) \oplus D(0)\bigr)$, where
$\Lambda^2 D(n)$ is the antisymmetric square of the spin-$n$
representation, and $\sigma(n) = \binom{2n+1}{2}$ counts the pairwise
couplings among $2n{+}1$ magnetic substates.

% ── 3 ──────────────────────────────────────────────────────────────────
\subsection*{3.\ Result 1: The Fine Structure Constant}

$\ainv$ is obtained from a cubic equation interpreted as a Dyson equation
$G^{-1} = G_0^{-1} - \Sigma(G)$ for the dressed inverse coupling:

\begin{equation*}
  \boxed{\ainv = B + \frac{1}{(n{+}2)\,B^2}}, \qquad
  B = \tau + \frac{\pi^2}{2\tau} - \frac{1}{2\tau^2}
    - \frac{1}{(n{-}1)\,\tau^3}
\end{equation*}

The base $B$ is a perturbative self-energy expansion on the BCC lattice
with coupling $g^2 = 1/2$ and Dirac multiplicity $N_D = 4$. The
coefficients have identified origins: $c_1 = \pi^2/2$ from the Brillouin
zone momentum integral $\langle|\mathbf{q}|^2\rangle_{\mathrm{BZ}}$;\
$c_2 = -d/\langle D^2\rangle = -1/2$ from the Dirac spectral moment;\
$c_3 = -1/(n{-}1) = -1/7$ from the spectral variance, with
$f_3(8) = -2/7$ as an exact algebraic identity.

\begin{table}[H]
\centering\small
\begin{tabular}{lc}
\toprule
Lattice framework prediction & $137.035\,999\,084$ \\
CODATA 2018 & $137.035\,999\,084(21)$ \\
Agreement & $< 0.005\ppb$ \\
\bottomrule
\end{tabular}
\end{table}

% ── 4 ──────────────────────────────────────────────────────────────────
\subsection*{4.\ Result 2: The Proton-Electron Mass Ratio}

From the same generator functions, with $\alpha$ computed self-consistently:

\begin{equation*}
  \boxed{\frac{m_p}{m_e} = \mu
    + \frac{\pi\mu\sigma\alpha^3}{4}
    + \frac{(2n{+}1)\,\pi^2\alpha(1 - \alpha^2)}{16}
    - \alpha^2}
\end{equation*}

The corrections decompose as a connection 1-form (vertex,
$\pi\mu\sigma\alpha^3/4 \approx 0.0762$), a curvature 2-form (vacuum
polarization, $(2n{+}1)\pi^2\alpha(1{-}\alpha^2)/16 \approx 0.0765$), and
a self-intersection ($-\alpha^2 \approx -5.3\times 10^{-5}$). The factor
$(1 - \alpha^2)$ is a dressed self-energy correction. The tree-level integer
$\mu(8) = 1836$ is $\tfrac{3}{2}\dim(\Lambda^2 D(8))\cdot(\rho(8))$ from
$\mathrm{SO}(3)$.

\begin{table}[H]
\centering\small
\begin{tabular}{lc}
\toprule
Lattice framework prediction & $1836.152\,674$ \\
CODATA 2018 & $1836.152\,673\,43(11)$ \\
Agreement & $< 0.03\ppb$ \\
\bottomrule
\end{tabular}
\end{table}

% ── 5 ──────────────────────────────────────────────────────────────────
\subsection*{5.\ Lattice Selection: Why BCC?}

Tested against SC ($n = 6$) and FCC ($n = 12$). The tree-level mass
generator $\mu(n)$ is the discriminator:

\begin{table}[H]
\centering\small
\begin{tabular}{cccccc}
\toprule
\textbf{Lattice} & $n$ & $\mu(n)$ & $\ainv$ \textbf{error}
  & \textbf{Mass error} & \textbf{Status} \\
\midrule
\rowcolor{bccgreen} BCC & 8 & 1836 & $< 0.005\ppb$ & $< 0.03\ppb$
  & Passes both \\
\rowcolor{scyellow} SC & 6 & 819 & $9.9\ppb$ & $-55.4\%$
  & Fails mass \\
\rowcolor{fccred} FCC & 12 & 5850 & $142.7\ppb\;(6.8\sigma)$ & $+218.6\%$
  & Fails both \\
\bottomrule
\end{tabular}
\end{table}

SC's $\ainv$ closeness is explained by $\langle D^2\rangle = 2d$ universality
for cubic lattices (first two loop corrections identical). Discrimination enters
at three loops via spectral variance and, decisively, through $\mu(n)$: no
perturbative correction bridges SC's 55\% mass deficit.

% ── 6 ──────────────────────────────────────────────────────────────────
\subsection*{6.\ Questions for the Reader}

\begin{enumerate}[label=(\alph*),nosep,leftmargin=1.5em]
  \item Is there a trivial or known reason why polynomials of 8 produce
    integers close to 137 and 1836?
  \item Does $g^2 = 1/2$, $N_D = 4$ on BCC correspond to a recognized
    lattice action?
  \item Is there a structural reason why the mass ratio should be sensitive
    to coordination number in this way?
  \item Is the SC/BCC near-degeneracy for $\ainv$
    (via $\langle D^2\rangle = 2d$) interesting or expected?
\end{enumerate}

\medskip
I would be grateful for any feedback, including identification of a trivial
explanation. Full paper (\LaTeX) available on request.

\vspace{6pt}
\hrule
\vspace{4pt}

% ── Appendix ───────────────────────────────────────────────────────────
\subsection*{Appendix: Verification Code (Python)}

\begin{Verbatim}[fontsize=\footnotesize]
import numpy as np
def tau(n): return n*(2*n+1)+1
def sigma(n): return n*(2*n+1)
def mu(n): return 1.5*sigma(n)*(n+1)
N=8; TAU,SIGMA,MU = tau(N),sigma(N),mu(N)
B = TAU + np.pi**2/(2*TAU) - 1/(2*TAU**2) - 1/((N-1)*TAU**3)
alpha_inv = B + 1/((N+2)*B**2); alpha = 1/alpha_inv
t1 = np.pi*MU*SIGMA*alpha**3/4
t2 = (2*N+1)*np.pi**2*alpha*(1-alpha**2)/16
mass = MU + t1 + t2 - alpha**2
print(f"alpha^-1 = {alpha_inv:.9f}")  # 137.035999084
print(f"m_p/m_e  = {mass:.6f}")       # 1836.152674
\end{Verbatim}

\vspace{4pt}
\hrule
\vspace{3pt}
\begin{center}
{\footnotesize\color{gray}
  Contact: \href{mailto:alan.javier.garcia@gmail.com}{alan.javier.garcia@gmail.com} \quad|\quad Full paper and working notes available on request}
\end{center}

\end{document}
