\documentclass[11pt,letterpaper]{article}

\usepackage[margin=0.9in]{geometry}
\usepackage{amsmath,amssymb}
\usepackage{booktabs}
\usepackage{array}
\usepackage{xcolor}
\usepackage{enumitem}
\usepackage{hyperref}
\usepackage{microtype}
\usepackage{fancyvrb}
\usepackage{float}
\usepackage{colortbl}

\hypersetup{colorlinks=true, linkcolor=blue!60!black, urlcolor=blue!60!black}

\definecolor{bccgreen}{HTML}{e8f5e8}
\definecolor{scyellow}{HTML}{fff8e0}
\definecolor{fccred}{HTML}{fde8e8}
\definecolor{headbg}{HTML}{e8e8f0}
\definecolor{darkblue}{HTML}{1a1a2e}
\definecolor{treecol}{HTML}{1a5276}
\definecolor{vtxcol}{HTML}{af601a}
\definecolor{vpcol}{HTML}{117864}
\definecolor{sicol}{HTML}{a93226}

\pagestyle{empty}

\newcommand{\ainv}{\alpha^{-1}}
\newcommand{\ppb}{\,\mathrm{ppb}}
\newcommand{\ppm}{\,\mathrm{ppm}}

\setlength{\parskip}{4pt}
\setlength{\parindent}{0pt}

\begin{document}

% -- Title --
\begin{center}
  {\Large\bfseries\color{darkblue}
    Numerical Coincidences from a BCC Lattice Framework}\\[4pt]
  {\itshape A concise summary for discussion --- not a claim of proof}\\[2pt]
  {\small Alan Garcia \quad|\quad February 2026 \quad|\quad Independent investigation}
\end{center}
\vspace{-2pt}
\hrule
\vspace{6pt}

% -- 1 --
\subsection*{1.\ What This Is}

I am an independent researcher exploring a lattice field theory framework
in which the fine structure constant, the proton-electron mass ratio, the
muon-electron mass ratio, the tau-electron mass ratio, the Higgs boson
mass, and the neutron-proton mass difference emerge from the spectral geometry
of the BCC lattice Dirac operator. The framework takes two inputs---the BCC
coordination number $n = 8$ and the transcendental $\pi$---with $d = 3$
(spatial dimensions) entering the muon, tau, Higgs, and neutron formulas,
and produces values for all six constants with zero free parameters. The Higgs
prediction (125.108~GeV) is genuinely predictive---it selects the ATLAS value
over CMS and is falsifiable at the HL-LHC. I am writing to ask whether you
can see a trivial explanation for the coincidences below, or whether the
lattice selection mechanism has structural content.

% -- 2 --
\subsection*{2.\ Generator Functions from $\mathrm{SO}(3)$ Representation Theory}

Four functions of the coordination number $n$ arise from the
representation theory of $\mathrm{SO}(3)$:

\begin{equation*}
  \tau(n) = n(2n{+}1) + 1 \qquad
  \sigma(n) = n(2n{+}1) \qquad
  \rho(n) = n + 1 \qquad
  \mu(n) = \tfrac{3}{2}\,\sigma\,\rho
\end{equation*}

At $n = 8$: $\tau = 137$, $\sigma = 136$, $\rho = 9$, $\mu = 1836$.
Here $\tau(n) = \dim\bigl(\Lambda^2 D(n) \oplus D(0)\bigr)$, where
$\Lambda^2 D(n)$ is the antisymmetric square of the spin-$n$
representation, and $\sigma(n) = \binom{2n+1}{2}$ counts the pairwise
couplings among $2n{+}1$ magnetic substates.

% -- 3 --
\subsection*{3.\ Correction Taxonomy}

All six BSM constants share a four-channel correction structure.
The color coding below identifies each channel across all formulas:

\begin{center}
\small
\renewcommand{\arraystretch}{1.1}
\begin{tabular}{>{\raggedright}p{1.8cm}p{5.5cm}}
\toprule
\textbf{Channel} & \textbf{Role} \\
\midrule
\textcolor{treecol}{\bfseries Tree} &
  Dominant term from lattice generators ($\tau$, $\mu$, $\sigma{-}d$, $\binom{n-1}{d}$) \\
\textcolor{vtxcol}{\bfseries Vertex} &
  Multiplicative dressing of tree level \\
\textcolor{vpcol}{\bfseries VP} &
  Vacuum polarization (spectral / loop correction) \\
\textcolor{sicol}{\bfseries Self-int.} &
  Universal self-intersection ($-\alpha^2$) or Dyson self-consistency \\
\bottomrule
\end{tabular}
\end{center}

\noindent\textbf{Algebraic forms} (color brackets mark each correction channel):
\begin{align*}
\ainv &=
  \textcolor{treecol}{\bigl[}\tau\textcolor{treecol}{\bigr]}
  + \textcolor{vpcol}{\bigl[}\tfrac{c_1}{\tau}
    - \tfrac{1}{2\tau^2}
    - \tfrac{1}{(n{-}1)\tau^3}
    + \tfrac{c_4}{\tau^4}\textcolor{vpcol}{\bigr]}
  + \textcolor{sicol}{\bigl[}\tfrac{1}{(n{+}2)B^2}\textcolor{sicol}{\bigr]}
\\[6pt]
\frac{m_p}{m_e} &=
  \textcolor{treecol}{\bigl[}\mu\textcolor{treecol}{\bigr]}
  \cdot\!\bigl(1 + \textcolor{vtxcol}{\bigl[}2(2n{+}1)\pi\alpha^3\textcolor{vtxcol}{\bigr]}\bigr)
  + \textcolor{vpcol}{\bigl[}\bigl(1{+}\tfrac{1}{2n}\bigr)\pi^2\alpha(1{-}\alpha^2)\textcolor{vpcol}{\bigr]}
  \;\textcolor{sicol}{\bigl[}{-}\alpha^2\textcolor{sicol}{\bigr]}
\\[6pt]
\frac{m_\mu}{m_e} &=
  \textcolor{treecol}{\bigl[}\tfrac{d}{2}\tau + \tfrac{c_1}{4}\textcolor{treecol}{\bigr]}
  + \textcolor{vpcol}{\bigl[}\tfrac{d}{2}\pi\alpha(1{-}\alpha^2)\textcolor{vpcol}{\bigr]}
  + \textcolor{vtxcol}{\bigl[}\tfrac{d}{2}\pi\sigma\alpha^3\textcolor{vtxcol}{\bigr]}
  \;\textcolor{sicol}{\bigl[}{-}\alpha^2\textcolor{sicol}{\bigr]}
\\[6pt]
Q &=
  \textcolor{treecol}{\bigl[}\tfrac{d{-}1}{d}\textcolor{treecol}{\bigr]}
  + \textcolor{vpcol}{\bigl[}\tfrac{(d{-}1)\pi^2\alpha^2}{(n{-}1)\sigma}\textcolor{vpcol}{\bigr]}
  \;\;\longrightarrow\; m_\tau/m_e \;\text{via Koide quadratic}
\\[6pt]
\frac{m_H}{m_p} &=
  \textcolor{treecol}{\bigl[}\sigma{-}d\textcolor{treecol}{\bigr]}
  \cdot\!\bigl(1 + \textcolor{vpcol}{\bigl[}\pi\alpha/\rho\textcolor{vpcol}{\bigr]}\bigr)
\\[6pt]
\frac{\Delta m}{m_e} &=
  \Bigl[\textcolor{treecol}{\bigl[}\tbinom{n{-}1}{d}\pi^2\alpha\textcolor{treecol}{\bigr]}
  \cdot\!\bigl(1 + \textcolor{vtxcol}{\bigl[}\tfrac{(n{-}d)\alpha}{\rho}\textcolor{vtxcol}{\bigr]}\bigr)
  \;\textcolor{sicol}{\bigl[}{-}\alpha^2\textcolor{sicol}{\bigr]}\Bigr]
  \times\!\bigl(1 + \textcolor{vpcol}{\bigl[}\tfrac{(n{-}1)\alpha^2}{n{+}2}\textcolor{vpcol}{\bigr]}\bigr)
\end{align*}

The uniformity is structural: every mass formula is built from the same
generators ($\tau$, $\sigma$, $\rho$, $\mu$), the same correction channels, and the
same coupling $\alpha$.  The \textcolor{vpcol}{VP} channel uses
$(1{-}\alpha^2)$ for absolute masses (proton, muon) but not for ratios or
differences. The \textcolor{sicol}{self-intersection} $-\alpha^2$ is universal
wherever the defect has a self-energy.

\medskip
\noindent\textbf{Complex and series representations:}

\begin{table}[H]
\centering\small
\renewcommand{\arraystretch}{1.15}
\begin{tabular}{lp{4.6cm}p{4.6cm}}
\toprule
\textbf{Constant} & \textbf{Complex domain} & \textbf{Series form} \\
\midrule
$\ainv$ &
  Cubic $z^3{-}Bz^2=1/(n{+}2)$: one real root, complex pair $|z_{2,3}|\sim\!\sqrt{\alpha}$ &
  Laurent series in $1/\tau$, convergent for $|1/\tau|<1$ \\
$m_p/m_e$ &
  VP zeros at $\alpha = \pm 1$; unitary for $|\alpha|<1$ &
  Polynomial in $\alpha$ (degree~3) \\
$m_\mu/m_e$ &
  Same $(1{-}\alpha^2)$ spectral weight (universal) &
  Polynomial in $\alpha$ (degree~3) \\
$m_\tau/m_e$ &
  $|Z|^2 = (3Q{-}1)/2$; discriminant $D \approx 139 > 0$ &
  DFT: $Z = \sum_k s_k\,\omega^k$, $\omega = e^{2\pi i/3}$ \\
$m_H/m_p$ &
  Integer gap $\sigma{-}d$ prevents level crossing &
  Taylor in $\pi\alpha/\rho$ (1st order) \\
$\Delta m/m_e$ &
  Landau-like pole at $\alpha \approx 1.20$, far from physical &
  Geometric: $\sum [(n{-}1)\alpha^2/(n{+}2)]^k$ \\
\bottomrule
\end{tabular}
\end{table}

\noindent In the complex domain, each formula enforces its own consistency
bound: the physical $\alpha \approx 1/137$ lies far from every pole,
branch point, and sign change, confirming that all six predictions are
deep in the perturbative regime.

% -- 4 --
\subsection*{4.\ Result 1: The Fine Structure Constant}

$\ainv$ is obtained from a cubic equation interpreted as a Dyson equation
$G^{-1} = G_0^{-1} - \Sigma(G)$ for the dressed inverse coupling:

\begin{equation*}
  \boxed{\ainv = B + \frac{1}{(n{+}2)\,B^2}}, \qquad
  B = \tau + \frac{c_1}{\tau} - \frac{1}{2\tau^2}
    - \frac{1}{(n{-}1)\,\tau^3}
    + \frac{c_4}{\tau^4}
\end{equation*}

The base $B$ is a perturbative self-energy expansion on the BCC lattice
with coupling $g^2 = 1/2$ and Dirac multiplicity $N_D = 4$. The
coefficients have identified spectral origins:
\begin{itemize}[nosep,leftmargin=1.5em]
  \item $c_1 = \pi^2/2$ \quad Brillouin zone momentum integral
    $\langle|\mathbf{q}|^2\rangle_{\mathrm{BZ}}$
  \item $c_2 = -\tfrac{1}{2}$ \quad Spectral mean:
    $-d/\langle D^2\rangle$
  \item $c_3 = -1/(n{-}1) = -\tfrac{1}{7}$ \quad Spectral variance
  \item $c_4 = \tfrac{2\rho}{2n{+}1}\,\pi^3 = \tfrac{18}{17}\,\pi^3$
    \quad Spectral skewness $\times$ zone integral
\end{itemize}

\begin{table}[H]
\centering\small
\begin{tabular}{lc}
\toprule
Lattice framework prediction & $137.035\,999\,177$ \\
CODATA 2022 & $137.035\,999\,177(21)$ \\
Agreement & $< 0.001\ppb$ \\
\bottomrule
\end{tabular}
\end{table}

\noindent\textit{Complex form.}  The Dyson equation is the cubic
$z^3 - Bz^2 - 1/(n{+}2) = 0$, with discriminant $\Delta < 0$: one real root
$z_1 = \ainv$ and a complex conjugate pair
$z_{2,3} \approx \pm\,0.027\,i$.  By Vieta's formulas,
$|z_{2,3}|^2 = 1/\bigl((n{+}2)\,\ainv\bigr) = \alpha/(n{+}2)$, so the
complex roots have modulus $\sim\!\sqrt{\alpha}$.
The physical coupling is the unique positive real root, separated from the
complex pair by $\ainv/|z_2| \approx 5\,000$; perturbative stability follows
from the complex roots being $O(\sqrt{\alpha})$ while the physical root is
$O(1/\alpha)$.

% -- 5 --
\subsection*{5.\ Result 2: The Proton-Electron Mass Ratio}

From the same generator functions, with $\alpha$ computed self-consistently.
The formula appears naturally in $n$-explicit factored form:

\begin{equation*}
  \boxed{\frac{m_p}{m_e}
    = \mu\bigl(1 + 2(2n{+}1)\,\pi\alpha^3\bigr)
    + \Bigl(1 + \frac{1}{2n}\Bigr)\pi^2\alpha(1{-}\alpha^2)
    - \alpha^2}
\end{equation*}

The denominators in the original notation ($\pi\mu\sigma\alpha^3/4$ and
$(2n{+}1)\pi^2\alpha/16$) are functions of the coordination number:
$4 = n/2$ and $16 = 2n$ at $n = 8$. The corrections decompose as:
\begin{itemize}[nosep,leftmargin=1.5em]
  \item \textbf{Vertex:} $\mu$ is dressed by a factor
    $1 + 2(2n{+}1)\pi\alpha^3$, where $2(2n{+}1) = 2\dim D(n)$.
  \item \textbf{VP:} A universal piece $\pi^2\alpha(1{-}\alpha^2)$ plus
    a lattice correction of relative size $1/(2n)$, vanishing as
    $n \to \infty$ (continuum limit).
  \item \textbf{Self-intersection:} $-\alpha^2$, universal.
\end{itemize}

\begin{table}[H]
\centering\small
\begin{tabular}{lc}
\toprule
Lattice framework prediction & $1836.152\,673\,5$ \\
CODATA 2022 & $1836.152\,673\,426(32)$ \\
Agreement & $< 0.03\ppb$ \\
\bottomrule
\end{tabular}
\end{table}

\noindent\textit{Complex form.}  In the complex $\alpha$-plane, the VP factor
$\alpha(1 - \alpha^2)$ has zeros at $\alpha = 0, \pm 1$.  The physical
$\alpha \approx 1/137$ lies deep in the convergent region $|\alpha| < 1$;
the boundary $|\alpha| = 1$ marks a phase transition where the VP changes
sign.  The $(1 - \alpha^2)$ envelope ensures the dressed propagator is
unitary for all $|\alpha| < 1$.

% -- 6 --
\subsection*{6.\ Result 3: The Muon-Electron Mass Ratio}

The muon is modeled as a binary defect excitation (two-node) on the same
lattice, in contrast to the proton's ternary defect (three-node). Its
formula follows the same correction taxonomy but with $d$ (spatial
dimension) replacing $n$ (coordination number) as the structural
parameter:

\begin{equation*}
  \boxed{\frac{m_\mu}{m_e}
    = \frac{d}{2}\bigl(\tau + \pi\alpha(1{-}\alpha^2)\bigr)
    + \frac{c_1}{4}
    + \frac{d}{2}\,\pi\sigma\alpha^3
    - \alpha^2}
\end{equation*}

The proton's vertex prefactor is $2/n$ (per-bond weight on a lattice with
$n$ neighbors); the muon's is $d/2$ (dimensional weight in $d$-dimensional
space). The muon vertex coefficient $(d/2)\,\pi\,\sigma$ equals the proton
vertex coefficient times $2d/\mu = 6/1836 = 1/306$.

\begin{table}[H]
\centering\small
\begin{tabular}{lc}
\toprule
Lattice framework prediction & $206.768\,282\,5$ \\
CODATA 2022 & $206.768\,282\,7(46)$ \\
Agreement & $1.1\ppb$ \quad ($0.05\sigma$) \\
\bottomrule
\end{tabular}
\end{table}

\noindent\textit{Complex form.}  The same $(1 - \alpha^2)$ spectral weight
governs the muon VP as the proton's, confirming its universality: the
analytic structure in the complex $\alpha$-plane is independent of defect
type (binary vs.\ ternary), depending only on the lattice coupling.

% -- 7 --
\subsection*{7.\ Result 4: The Tau-Electron Mass Ratio}

The tau lepton mass is predicted by dressing the Koide relation. Yoshio
Koide (1982) observed empirically that
$(m_e + m_\mu + m_\tau)/(\sqrt{m_e} + \sqrt{m_\mu} + \sqrt{m_\tau})^2
\approx 2/3$.  In the lattice framework, the binary defect is a
1-dimensional line segment in $d$~spatial dimensions; its normal bundle
has fiber dimension $d{-}1 = 2$, giving a transverse fraction
$(d{-}1)/d = 2/3$.  This tree-level constant receives a radiative
correction from the same spectral ingredients that enter the $\alpha$
formula:

\begin{equation*}
  \boxed{Q = \frac{d{-}1}{d}
    + \frac{(d{-}1)\,\pi^2\alpha^2}{(n{-}1)\,\sigma}}, \qquad
  \frac{m_e + m_\mu + m_\tau}
       {(\sqrt{m_e} + \sqrt{m_\mu} + \sqrt{m_\tau})^{\,2}} = Q
\end{equation*}

The correction $\delta Q = (d{-}1)\pi^2\alpha^2/\!\bigl((n{-}1)\sigma\bigr)$
decomposes as:
\begin{itemize}[nosep,leftmargin=1.5em]
  \item $(d{-}1)$ --- transverse codimension (number of normal directions)
  \item $\pi^2$ --- BZ momentum integral (same $\pi^2$ as $c_1$ in Result~4)
  \item $\alpha^2$ --- two-loop EM coupling
  \item $1/(n{-}1) = 1/7$ --- spectral variance (same as $c_3$ in Result~4)
  \item $1/\sigma = 1/136$ --- pairwise channel normalization
\end{itemize}

With $m_e = 1$ and $m_\mu$ from Result~6, the Koide equation reduces to a
quadratic in $x = \sqrt{m_\tau}$ with closed-form solution:
\begin{equation*}
  x = \frac{Q(1 + \sqrt{m_\mu}\,) + \sqrt{(1 + m_\mu)(2Q{-}1) + 2Q\sqrt{m_\mu}}}{1 - Q}\,,
  \qquad m_\tau = x^2
\end{equation*}

\begin{table}[H]
\centering\small
\begin{tabular}{lc}
\toprule
Lattice framework prediction & $3477.4799$ \\
CODATA 2022 & $3477.48 \pm 0.57$ \\
Agreement & $0.027\,\mathrm{ppm}$ \quad ($0.0002\sigma$) \\
\bottomrule
\end{tabular}
\end{table}

\noindent\textit{Note:} The bare Koide constant $Q = 2/3$ gives $m_\tau/m_e = 3477.44$
(11\,ppm, $0.07\sigma$).  The dressed $Q$ improves this by a factor of~415.
The correction was identified by systematic numerical scan and interpreted
geometrically as the leading Seeley--DeWitt heat kernel coefficient of the
transverse Laplacian on the binary defect; a first-principles derivation from
the lattice Lagrangian remains an open problem.

\noindent\textit{Complex form.}  Define
$Z = \sum_k (\sqrt{m_k}/S)\,\omega^k$ with $\omega = e^{2\pi i/3}$ and
$S = \sum_k \sqrt{m_k}$.  Then $|Z|^2 = (3Q - 1)/2$.  At tree level
($Q = 2/3$), $|Z| = 1/\sqrt{2}$ exactly; the BSM correction shifts $|Z|^2$
by $3\delta Q/2 \approx 1.7 \times 10^{-6}$.  The phase $\arg Z$ encodes
the mass hierarchy, while the BSM framework fixes only the modulus.  The
quadratic discriminant $D \approx 138.8 \gg 0$ confirms real roots; a
complex tau mass would require $Q < 0.50$, far below the physical
$Q \approx 2/3$, placing the framework deep within the unitary domain.

% -- 8 --
\subsection*{8.\ Result 5: The Higgs Boson Mass}

The Higgs boson is modeled as the \emph{breathing mode} of the lattice---the
uniform oscillation of lattice spacing about equilibrium. Each site has
$\sigma = n(2n{+}1) = 136$ pairwise scalar modes; $d = 3$ become Goldstone
bosons (eaten by $W^\pm$, $Z$), leaving $\sigma - d = 133$ physical scalar
channels. The Higgs mass counts these surviving channels in proton-mass
units, with a universal vacuum polarization correction:

\begin{equation*}
  \boxed{\frac{m_H}{m_p}
    = (\sigma - d)\!\left(1 + \frac{\pi\alpha}{\rho}\right)}
\end{equation*}

where $\pi\alpha$ is the one-loop VP factor and $\rho = n{+}1 = 9$ counts
the radial shells in the spin-$n$ representation. The tree level gives
$133 \times 0.9383 = 124.79$~GeV (0.26\% low); the correction raises this
to:

\begin{equation*}
  m_H^{\mathrm{BSM}} = 133.339 \times 0.938272\,\mathrm{GeV}
    = 125.108\,\mathrm{GeV}
\end{equation*}

\begin{table}[H]
\centering\small
\begin{tabular}{lccc}
\toprule
\textbf{Experiment} & \textbf{Measured (GeV)} & \textbf{BSM (GeV)} & \textbf{Deviation} \\
\midrule
ATLAS (combined) & $125.11 \pm 0.11$ & $125.108$ & $0.02\sigma$ \\
CMS (combined) & $125.35 \pm 0.15$ & $125.108$ & $1.6\sigma$ \\
\bottomrule
\end{tabular}
\end{table}

\noindent The BSM prediction matches ATLAS to $0.02\sigma$ but is $1.6\sigma$
from CMS---the framework \emph{picks a side} in the ATLAS/CMS tension
($1.3\sigma$). The HL-LHC, projecting 21~MeV precision by 2041, will
confirm or rule out this prediction: convergence toward 125.11~GeV confirms
it ($0.1\sigma$); convergence toward 125.35~GeV rules it out ($11.5\sigma$).
This is the first BSM result that is more precise than experiment and
genuinely predictive rather than postdictive.

\noindent\textit{Complex form.}  In the complex mass-squared plane, the
$\sigma = 136$ scalar modes split into $d = 3$ massless Goldstone poles (at
$m^2 = 0$, eaten by $W^\pm$, $Z$) and $\sigma - d = 133$ massive poles near
$m^2 = m_H^2$.  The integer gap $\sigma - d$ between sectors prevents level
crossing under the VP correction $\pi\alpha/\rho$, which shifts all 133 poles
uniformly.  The integrality of the Goldstone count ($d = 3$ spatial
generators) is the complex-analytic analogue of the Goldstone theorem.

% -- 9 --
\subsection*{9.\ Result 6: The Neutron-Proton Mass Difference}

The neutron-proton mass difference $\Delta m = m_n - m_p = 1.29334\;\mathrm{MeV}
= 2.531\,030\;m_e$ is an $O(1)$ quantity in electron mass units---not a
perturbative correction. In the BSM framework, the proton and neutron are
the same ternary defect differing in \emph{orientation}: how the triangle
sits among the $\binom{n-1}{d} = \binom{7}{3} = 35$ possible spatial triads.

\begin{equation*}
  \boxed{\frac{\Delta m}{m_e}
    = \left[\binom{n{-}1}{d}\,\pi^2\alpha\!\left(1 + \frac{(n{-}d)\,\alpha}{\rho}\right)
    - \alpha^2\right]
    \times
    \left(1 + \frac{(n{-}1)\,\alpha^2}{n{+}2}\right)}
\end{equation*}

The corrections follow the universal BSM taxonomy:
\begin{itemize}[nosep,leftmargin=1.5em]
  \item \textbf{Tree:} $\binom{7}{3}\pi^2\alpha = 35\,\pi^2\alpha = 2.5208\;m_e$
    (orientational modes $\times$ EM lattice coupling). 0.41\% below experiment.
  \item \textbf{Vertex:} $(n{-}d)\alpha/\rho = 5\alpha/9$ (spectator bonds /
    radial modes). Closes to 37\,ppm.
  \item \textbf{Self-intersection:} $-\alpha^2$, universal.
  \item \textbf{VP:} $\times(1 + (n{-}1)\alpha^2/(n{+}2)) = \times(1 + 7\alpha^2/10)$,
    multiplicative two-loop correction. The prefactor $(n{-}1)/(n{+}2) = 7/10$
    is the aspect ratio of $\binom{n-1}{d} = (n{-}1)(n{+}2)/2$. Closes to
    0.036\,ppm.
\end{itemize}

\begin{table}[H]
\centering\small
\begin{tabular}{lc}
\toprule
Lattice framework prediction & $2.531\,029\,91$ \\
Experiment & $2.531\,030\,00(3)$ \\
Agreement & $0.036\,\mathrm{ppm}$ \quad ($0.03\sigma$) \\
\bottomrule
\end{tabular}
\end{table}

\noindent The binomial coefficient $\binom{n-1}{d}$ is the sharpest lattice
discriminator: SC ($n = 6$) gives $\binom{5}{3} = 10$, predicting 50\% of
experiment; FCC ($n = 12$) gives $\binom{11}{3} = 165$, predicting 114\%
above experiment. Only BCC works.

\noindent\textit{Complex form.}  The multiplicative VP resums a geometric
series $1/(1 - (n{-}1)\alpha^2/(n{+}2))$ with a pole at
$\alpha_{\mathrm{pole}} = \sqrt{(n{+}2)/(n{-}1)} = \sqrt{10/7} \approx 1.20$
in the complex $\alpha$-plane.  The physical $\alpha = 1/137$ gives a
convergence ratio of $3.7 \times 10^{-5}$; truncation to first order
introduces an error of $O(\alpha^4) \sim 10^{-9}\;m_e$, well below all other
corrections.  The Landau-like pole lies in the strong-coupling regime, far
from the perturbative domain.

% -- 10 --
\subsection*{10.\ The Vacuum Polarization Catalog}

Every BSM constant has a classified vacuum polarization correction:

\begin{table}[H]
\centering\small
\begin{tabular}{lllc}
\toprule
\textbf{Observable} & \textbf{VP correction} & \textbf{Type} & \textbf{Loop order} \\
\midrule
$\ainv$ & $\pi^2/(2\tau)$ & Additive & 0 (BZ integral) \\
$m_p/m_e$ & $(1+1/(2n))\pi^2\alpha(1{-}\alpha^2)$ & Additive & 1-loop \\
$m_\mu/m_e$ & $(d/2)\pi\alpha(1{-}\alpha^2)$ & Additive & 1-loop \\
$m_H/m_p$ & $\pi\alpha/\rho$ & Multiplicative & 1-loop \\
$\Delta m/m_e$ & $(n{-}1)\alpha^2/(n{+}2)$ & Multiplicative & 2-loop \\
\bottomrule
\end{tabular}
\end{table}

\noindent Key patterns: the $(1{-}\alpha^2)$ self-energy envelope appears for
\emph{absolute} masses (proton, muon) but cancels for \emph{ratios}
($m_H/m_p$) and \emph{differences} ($\Delta m$). Each VP sits one loop above
its tree level; the $\Delta m$ tree starts at $O(\pi^2\alpha)$, so its VP
starts at $O(\alpha^2)$ and carries no factor of $\pi$ because the Brillouin
zone integral is already absorbed into the tree.

% -- 11 --
\subsection*{11.\ Lattice Selection: Why BCC?}

Tested against SC ($n = 6$) and FCC ($n = 12$). The tree-level mass
generator $\mu(n)$ is the discriminator:

\begin{table}[H]
\centering\small
\begin{tabular}{ccccccc}
\toprule
\textbf{Lattice} & $n$ & $\mu(n)$ & $\binom{n-1}{d}$ & $\ainv$ \textbf{error}
  & \textbf{Mass error} & \textbf{Status} \\
\midrule
\rowcolor{bccgreen} BCC & 8 & 1836 & 35 & $< 0.001\ppb$ & $< 0.03\ppb$
  & Passes all \\
\rowcolor{scyellow} SC & 6 & 819 & 10 & $9.9\ppb$ & $-55.4\%$
  & Fails mass \\
\rowcolor{fccred} FCC & 12 & 5850 & 165 & $142.7\ppb\;(6.8\sigma)$ & $+218.6\%$
  & Fails both \\
\bottomrule
\end{tabular}
\end{table}

SC's $\ainv$ closeness is explained by $\langle D^2\rangle = 2d$ universality
for cubic lattices (first two loop corrections identical). Discrimination enters
at three loops via spectral variance and, decisively, through $\mu(n)$: no
perturbative correction bridges SC's 55\% mass deficit.

% -- 12 --
\subsection*{12.\ Questions for the Reader}

\begin{enumerate}[label=(\alph*),nosep,leftmargin=1.5em]
  \item Is there a trivial or known reason why polynomials of 8 produce
    integers close to 137 and 1836?
  \item Does $g^2 = 1/2$, $N_D = 4$ on BCC correspond to a recognized
    lattice action?
  \item The proton vertex denominator is $n/2$ and the VP denominator is
    $2n$. Does this $n$-dependence have precedent in lattice perturbation
    theory?
  \item The proton's structural parameter is $n$ (coordination); the
    muon's is $d$ (dimension). Does this topological/geometric distinction
    correspond to anything in defect field theory?
  \item Is the SC/BCC near-degeneracy for $\ainv$
    (via $\langle D^2\rangle = 2d$) interesting or expected?
  \item The dressed Koide correction uses $1/(n{-}1)$ (spectral
    variance) and $1/\sigma$ (pairwise channels)---the same quantities
    appearing in the $\ainv$ expansion. Is there a known mechanism by
    which a constraint on mass \emph{ratios} receives radiative
    corrections from the same spectral coefficients as a coupling?
  \item The interpretation $Q = (d{-}1)/d$ identifies the Koide constant
    with the transverse fraction of a line defect. Does this codimensional
    constraint have precedent in defect field theory or conformal field theory?
  \item The 35 orientational triads ($\binom{7}{3}$) provide a lattice
    analogue of isospin. Does the mapping to SU(2) isospin---two states
    from 35 modes---have a natural group-theoretic explanation?
  \item The $\Delta m$ VP prefactor $(n{-}1)/(n{+}2) = 7/10$ is the
    aspect ratio of the binomial coefficient $\binom{n-1}{d} = (n{-}1)(n{+}2)/2$.
    Is there a known mechanism by which a radiative correction reuses the
    combinatorial factors of its own tree level?
  \item Can the multiplicative VP $\times(1 + (n{-}1)\alpha^2/(n{+}2))$
    be derived from the isospin heat kernel on the ternary defect?
\end{enumerate}

\medskip
I would be grateful for any feedback, including identification of a trivial
explanation. Full paper (\LaTeX) available on request.

\vspace{6pt}
\hrule
\vspace{4pt}

% -- Appendix --
\subsection*{Appendix: Verification Code (Python)}

\begin{Verbatim}[fontsize=\footnotesize]
from mpmath import mp, mpf, pi, sqrt
from math import comb
mp.dps = 50
def tau(n): return n*(2*n+1)+1
def sigma(n): return n*(2*n+1)
def mu(n): return mpf(3)/2*sigma(n)*(n+1)
N = 8; d = 3
TAU, SIGMA, RHO, MU = mpf(tau(N)), mpf(sigma(N)), mpf(N+1), mu(N)
c1 = pi**2 / 2
c4 = (2*RHO/(2*N+1)) * pi**3
B = TAU + c1/TAU - 1/(2*TAU**2) - 1/((N-1)*TAU**3) + c4/TAU**4
alpha_inv = B + 1/((N+2)*B**2); alpha = 1/alpha_inv
mass = MU*(1+2*(2*N+1)*pi*alpha**3) \
       + (1+1/(2*N))*pi**2*alpha*(1-alpha**2) - alpha**2
muon = mpf(d)/2*(TAU+pi*alpha*(1-alpha**2)) \
       + c1/4 + mpf(d)/2*pi*SIGMA*alpha**3 - alpha**2
Q = mpf(d-1)/d + mpf(d-1)*pi**2*alpha**2/((N-1)*SIGMA)
b = sqrt(muon)
D = (1 + muon)*(2*Q - 1) + 2*Q*b
tau_mass = ((Q*(1+b) + sqrt(D)) / (1-Q))**2
higgs = (SIGMA - d) * (1 + pi*alpha/RHO)
dm_tree = comb(N-1, d) * pi**2 * alpha
dm_prevp = dm_tree*(1 + (N-d)*alpha/RHO) - alpha**2
dm = dm_prevp * (1 + (N-1)*alpha**2/(N+2))
print(f"alpha^-1  = {alpha_inv}")   # 137.035999177...
print(f"m_p/m_e   = {mass}")        # 1836.15267348...
print(f"m_mu/m_e  = {muon}")        # 206.768282475...
print(f"m_tau/m_e = {tau_mass}")    # 3477.47990762...
print(f"m_H/m_p   = {higgs}")       # 133.339 (125.108 GeV)
print(f"Dm/m_e    = {dm}")          # 2.53102991 (expt: 2.53103000(3))
\end{Verbatim}

\vspace{4pt}
\hrule
\vspace{3pt}
\begin{center}
{\footnotesize\color{gray}
  Contact: \href{mailto:alan.javier.garcia@gmail.com}{alan.javier.garcia@gmail.com} \quad|\quad Full paper and working notes available on request}
\end{center}

\end{document}
